\documentclass{article}
\usepackage{amsmath}
\usepackage{amsfonts}
\usepackage[ddmmyyyy]{datetime2} 

\begin{document}

\title{Modelo Matemático - Problema Transporte de Grãos Donovan T\&T}
\author{Graciele Rodrigues}
\date{22/09/2024}
\maketitle


\section*{Problema}

A Donovan T\&T é uma transportadora sediada em Oakwood, nos Estados Unidos. A empresa é especializada no transporte de grãos (milho, soja, feijão, etc.) por caminhões. Recentemente, a diretoria percebeu que sua receita está estagnada, e estuda alternativas para aumentar o faturamento. Uma delas é garantir as melhores escolhas na hora de planejar o transporte a ser realizado por um caminhão. Ajude a empresa nessa tarefa!

O caminhão possui uma capacidade de transporte (em t) e um volume máximo (em m3). Há diferentes tipos de grãos a serem transportados. Cada tipo possui uma densidade (em t/m3), um volume máximo que pode ser transportado (em m3), definido pela legislação do estado, e uma receita esperada (em \$/m3).

Construa um modelo de programação linear para decidir quanto de cada grão transportar (m3), visando aumentar a receita da empresa.

\subsection*{Modelo Matemático}

\textbf{Dados:}
\begin{itemize}
    \item $G$: Conjunto de grãos.
    \item $V_g$: Volume máximo de cada grão $g \in G$ (m³).
    \item $D_g$: Densidade de cada grão $g \in G$ (t/m³).
    \item $R_g$: Receita esperada de cada grão $g \in G$ (\$/m³).
    \item $V_{max}$: Volume total disponível no caminhão (m³).
    \item $T_{max}$: Capacidade total do caminhão (t).
\end{itemize}

\textbf{Variáveis de decisão:}
\begin{itemize}
    \item $x_g$, $\forall g \in G$: Volume de cada grão $g \in G$ a ser transportado (m³).
\end{itemize}

\textbf{Função Objetivo:}
\[
\text{maximiza} \quad z = \sum_{g \in G} R_g \cdot x_g
\]

\textbf{Restrições:}
\begin{align*}
    &\sum_{g \in G} x_g \leq V_{max} \quad \text{(Capacidade de volume do caminhão não pode ser ultrapassada)} \\
    &\sum_{g \in G} D_g \cdot x_g \leq T_{max} \quad \text{(Capacidade de peso do caminhão não pode ser ultrapassada)} \\
    &x_g \leq V_g, \quad \forall g \in G \quad \text{(Limite de volume por grão não pode ser ultrapassado)} \\
    &x_g \geq 0, \quad \forall g \in G \quad \text{(Não negatividade)}
\end{align*}

\end{document}